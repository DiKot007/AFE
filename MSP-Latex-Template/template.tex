\documentclass{WileyMSP-template}
%%%% - удалить блок ниже перед отправкой
\usepackage[T1,T2A]{fontenc} % поддержка кириллических символов
\usepackage[utf8]{inputenc} % поддержка кодировки utf8
\usepackage[english,russian]{babel}  % Включаем пакет для поддержки русского языка (правильно ставятся переносы строк)  
\hbadness=2000 % отключили warning со значением менее 2000 (макс. 10000)
%%%% - удалить блок выше перед отправкой

\begin{document}


\pagestyle{fancy}
\rhead{\includegraphics[width=2.5cm]{vch-logo.png}}


\title{
	 Гистеризис поляризации поликристаллического антиферроэлектрика с наведенной кристаллографической анизотропией}

\maketitle



% Author: Please give full first and last names for authors and include * after the name of all corresponding authors

\author{Konstnatin V. Nefedev,}
\author{Author Two}
\author{Author Three*}



% Dedication

\dedication{Optional dedication here. If no dedication is required, please leave blank}



% Affiliations: Please provide adacemic titles (Prof. or Dr.) for all authors where applicable, and include an institutional email address for all corresponding authors
\begin{affiliations}
A. N. Author, A. N. O. Author\\
Address\\
Email Address: nefedev.kv@dvfu.ru

A. N. O. Author\\
Address Far Eastern Fedral University

\end{affiliations}


% Keywords: Please provide a minimum of three and a maximum of seven keywords, separated by commas

\keywords{Keyword 1, Keyword 2, Keyword 3}



% Abstract should be written in the present tense and impersonal style (i.e., avoid we), and be at most 200 words long
\begin{abstract}

Please insert your abstract here

\end{abstract}

% Text: Please use section headings and subheadings as specified below. For communications, all section headings apart from Experimental Section should be removed
% Please make the first reference to a display item bold: \textbf{Figure 1}
% Do not abbreviate Figure, Equation, etc.; display items are always singular, i.e., Figure 1 and 2.
% Equations are always singular, i.e., Equation 1 and 2, and should be inserted using the {equation} environment, not as graphics
% Please do not use footnotes in the text, additional information can be added to the Reference list.


\section{Введение}


The study of AgNbO3 antiferroelectric (AFE) ceramic for energy storage

AgNbO3 antiferroelectric (AFE) ceramics is a new type of lead-free energy storage medium with good application prospects in the field of high pulse power, but its wide electrical hysteresis leads to a decrease in energy storage efficiency and an increase in heat loss, which affects the reliability of the device. Relaxor AFE ceramics with slanted hysteresis loops can significantly improve energy storage performance, but there are few reports in AgNbO3 ceramic systems. Therefore, this project intends to take the AgNbO3-based ceramic system as the research object, explore the influence of powder raw materials and sintering process on the size and density of ceramic grains, and establish a technical approach for densification sintering of AgNbO3-based AFE ceramics. Then, by adjusting the valence state, ion polarizability or solid solution amount of the added metal ions, a relaxor AgNbO3-based AFE ceramic system with both micro/nano-scale domains is constructed. On this basis, the effects of the crystal structure, grain size and domain structure of different compositions on the polarization behavior will be studied, the interaction law between various factors affecting the electrical hysteresis behavior of materials will be explored, and finally the polarization mechanism of relaxor AgNbO3 AFE ceramics will be revealed. The results lay a theoretical and technical foundation for the application of AgNbO3-based AFE energy storage ceramics.

чтобы увеличить эффективность сохранения энергии, уменьшить потери, желательно получить однодоменный образец, который в нулевом поле будет находиться в конфигурации антиферроэлектрических атомарных нитей, полос. Но даже если зерно будет многодоменным, наведенная кристаллографическая анизотропия должна выровнять векторы внутренних полей, и срыв в поле должен быть резким, т.е. коэрцитивные свойства должны быть более выраженными. Я предполагаю, это можно сделать, если замораживать антиферроэлектрический материалы из жидкого расплава в сильном электрическом  поле. 

Задачи

1) Установить критический размер для условия однодоменности для антиферроэлектрического упорядочения в одном зерне

2) Выполнить расчет петли гистерезиса для одного зерна в модели Изинга с гамильтонианом из работы 
Misirlioglu I. B. et al. Antiferroelectric hysteresis loops with two exchange constants using the two dimensional Ising model //Applied Physics Letters. – 2007. – Т. 91. – №. 20.

3) Найти значения констант обменной энергии ферроэлектрического и антиферроэлектриеского упорядочения, константы взаимодействия с внешним полем, число Монте-Карло шагов, при которых петля имеет форму близкую к прямоугольной, т.е. максимальную площадь, при этом в нулевом внешнем электрическом поле площадь петли должна быть равна нулю (упорядочение антиферроэлектическое) 

4) Найти каким образом связаны число Монте-Карло шагов с термодинамическими флуктуациями, или время построения петли (шаг по полю).

5) Найти каким образом связаны значения констант обменной энергии ферроэлектрического и антиферроэлектриеского упорядочения, константы взаимодействия с внешним полем с решеткой материала, с элементами таблицы Менделеева.  

6) Сформировать поликристал из однодоменных зерен с рэндомизированными осями кристаллографической анизотропии, построить петлю гистерезиса поляризации

7) Сформировать поликристалл из однодоменных зерен с наведенной анизотропией кристаллографических осей, построить петлю гистерезиса поляризации  


\subsection{First Subsection}


\subsubsection{First Sub Subsection}


\threesubsection{First lowest-level subsection}


\section{Conclusion}

% Experimental section

\section{Experimental Section}
\threesubsection{First part of experimental section}\\
\threesubsection{Second part of experimental section}\\



\medskip
\textbf{Supporting Information} \par %Please delete the Suppporting Information statement if it is not applicable. Please supply Supporting Information in another file. Supporting information should not be provided in .tex format
Supporting Information is available from the Wiley Online Library or from the author.



% Acknowledgements
\medskip
\textbf{Acknowledgements} \par %delete if not applicable))
Please insert your acknowledgements here

% References
\medskip

% Use the following code if you wish to generate your bibliography with BibTeX;
% replace the string "MSP-template" below with the name(s) of
% the BibTeX data base(s) you want to use.
% The resulting bibliography-output (the content of the .bbl file)
% must be pasted back into this file before submission.
% Please also include your BibTeX data base file(s) in your submission
% so that we can re-run BibTeX if necessary.
%
%\bibliographystyle{MSP}
%\bibliography{MSP-template}

\textbf{References}\\

1	((Journal articles)) a) A. B. Author 1, C. D. Author 2, Adv. Mater. 2006, 18, 1; b) A. Author 1, B. Author 2, Adv. Funct. Mater. 2006, 16, 1.\\
2	((Work accepted)) A. B. Author 1, C. D. Author 2, Macromol. Rapid Commun., DOI: 10.1002/marc.DOI.\\
3	((Books)) H. R. Allcock, Introduction to Materials Chemistry, Wiley, Hoboken, NJ, USA 2008.\\
4	((Edited books or proceedings volumes)) J. W. Grate, G. C. Frye, in Sensors Update, Vol. 2 (Eds: H. Baltes, W. Göpel, J. Hesse), Wiley-VCH, Weinheim, Germany 1996, Ch. 2.\\
5	((Presentation at a conference, proceeding not published)) Author, presented at Abbrev. Conf. Title, Location of Conference, Date of Conference ((Month, Year)).\\
6	((Thesis)) Author, Degree Thesis, University (location if not obvious), Month, Year.\\
7	((Patents)) a) A. B. Author 1, C. D. Author 2 (Company), Country Patent Number, Year; b) W. Lehmann, H. Rinke (Bayer AG) Ger. 838217, 1952.\\
8	((Website)) Author, Short description or title, URL, accessed: Month, Year.\\
9	…((Please include all authors, and do not use “et al.”))\\




% Figures/tables and captions
% Permission statements are required for all figures reproduced or adapted from previously published articles/sources. Please also ensure that all necessary permissions to reproduce images have been received
% Please remove these statements for original figures


\begin{figure}
  \includegraphics[width=\linewidth]{placeholder-image.png}
  \caption{Figure 1 caption goes here. Reproduced with permission.\textsuperscript{[Ref.]} Copyright Year, Publisher. }
  \label{fig:boat1}
\end{figure}

\begin{figure}
  \includegraphics[width=\linewidth]{placeholder-image.png}
  \caption{Figure 2 caption goes here. Reproduced with permission.\textsuperscript{[Ref.]} Copyright Year, Publisher.}
  \label{fig:boat1}
\end{figure}

\begin{figure}
  \includegraphics[width=\linewidth]{placeholder-image.png}
  \caption{Figure 3 caption goes here. Reproduced with permission.\textsuperscript{[Ref.]} Copyright Year, Publisher.}
  \label{fig:boat1}
\end{figure}

\begin{table}
 \caption{Table 1 caption}
  \begin{tabular}[htbp]{@{}lll@{}}
    \hline
    Description 1 & Description 2 & Description 3 \\
    \hline
    Row 1, Col 1  & Row 1, Col 2  & Row 1, Col 3  \\
    Row 2, Col 1  & Row 2, Col 2  & Row 2, Col 3  \\
    \hline
  \end{tabular}
\end{table}


% Please provide Biographies and photos for Essays, Feature Articles, Progress Reports, Reviews, and Perspectives for those authors who should be highlighted  
% These should be at most 100 words long
% For other article types this section can be removed
% Photographs should be 40mm broad and 50 mm high

\begin{figure}
  \includegraphics{bio-placeholder.jpg}
  \caption*{Biography}
\end{figure}

\begin{figure}
  \includegraphics{bio-placeholder.jpg}
  \caption*{Biography}
\end{figure}

\begin{figure}
  \includegraphics{bio-placeholder.jpg}
  \caption*{Biography}
\end{figure}

\begin{figure}
  \includegraphics{bio-placeholder.jpg}
  \caption*{Biography}
\end{figure}


% Table of contents entry should be 50 - 60 words long
% Image should be 55 mm broad and 50 mm high or 110 mm broad and 20 mm high


\begin{figure}
\textbf{Table of Contents}\\
\medskip
  \includegraphics{toc-image.png}
  \medskip
  \caption*{ToC Entry}
\end{figure}


\end{document}
